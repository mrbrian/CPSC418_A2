\documentclass{assignment}
\usepackage[normalem]{ulem}
\usepackage{amsfonts}
\usepackage{amsmath}
\usepackage{intcalc}
\coursetitle{Introduction to Cryptography}
\courselabel{CPSC 418}
\exercisesheet{Home Work \#2}{}
\student{Brian Yee - 00993104}
\semester{Fall 2016}
\newcommand\tab[1][0.5cm]{\hspace*{#1}}
%\usepackage[pdftex]{graphicx}
%\usepackage{subfigure}

\begin{document}

\begin{center}
\renewcommand{\arraystretch}{2}
\begin{tabular}{|c|c|c|} \hline
Problem & Marks \\ \hline \hline
1 & \\ \hline
2 & \\ \hline
3 & \\ \hline
4 & \\ \hline
5 & \\ \hline
6 & \\ \hline
7 & \\ \hline \hline
Total & \\ \hline
\end{tabular}
\end{center}

\bigskip

\begin{problemlist}
\pbitem Conditional entropy
\begin{problem}
\begin{answer}
\\
1.a)\\
$H(M|C)=\sum_{c\in C}p(C)\sum_{m\in M}p(M|C)log_2(\frac{1}{p(M|C)})$\\
$H(M|C)=\sum p(C)\sum p(M|C)log_2(\frac{1}{p(M|C)})$\\
%$=\frac{1}{4}(\frac{1}{2}log_2(2) + \frac{1}{2}log_2(2)) + \frac{1}{4}(\frac{1}{2}log_2(2) + \frac{1}{2}log_2(2)) + \frac{1}{4}(\frac{1}{2}log_2(2) + \frac{1}{2}log_2(2)) + \frac{1}{4}(\frac{1}{2}log_2(2) + \frac{1}{2}log_2(2))$\\
$=4*\frac{1}{4}(\frac{1}{2}log_2(2)+ \frac{1}{2}log_2(2))$\\
$=1$\\
\\
1.b)\\
$H(M|C)=\sum p(C)\sum p(M|C)log_2(\frac{1}{p(M|C)})$\\

Since the cryptosystem provides perfect secrecy, $p(M|C)=p(M)$.\\

$=\sum p(C)\sum p(M)log_2(\frac{1}{p(M)})$\\

We know $\sum p(M)log_2(\frac{1}{p(M)})=log_2(\frac{1}{p(M)})$, when a cryptosystem provides perfect secrecy. \\

$=\sum p(C)log_2(\frac{1}{p(M)})$\\
\\
With perfect secrecy, every M is equiprobable, so every C is equiprobable.
Since $|C| = |M|$  (as every unique C comes from encrypting some unique M), so we have
p(C) = p(M).\\
\\
Thus,\\

$=\sum p(M)log_2(\frac{1}{p(M)})$\\
$=H(M)$\\
\\
1.c)\\
No, since $p(M|C)=\frac{1}{2}\neq\frac{1}{4}=p(M)$.\\
\\
\end{answer}
\end{problem}

\pbitem Binary polynomial arithmetic
\begin{problem}
\begin{answer}
\\
2.a.i)\\
$x^3$\\
$x^3+1$\\
$x^3+x$\\
$x^3+x+1$\\
$x^3+x^2$\\
$x^3+x^2+1$\\
$x^3+x^2+x$\\
$x^3+x^2+x+1$\\
\\
2.a.ii)\\
$x^3=x*x*x$\\
$x^3+1=(x+1)(x^2-x+1)$\\
$x^3+x=x(x^2+x)$\\
$x^3+x+1=$ irreducible\\
$x^3+x^2=x^2(x+1)$\\
$x^3+x^2+1=$ irreducible\\
$x^3+x^2+x=x(x^2+x+1)$\\
$x^3+x^2+x+1=(x+1)(x^2+1)$\\
\\
2.a.iii)\\
Let $A(x)$ be a degree 3 polynomial.  If $A(x)$ is reducible, then it must the product of a degree 1 polynomial and some other polynomial(s) (of degree 2 or 1).  In either case, there is a polynomial of degree 1 as a factor.  The two possible polynomials of degree 1 are $P_1=x+1$ and $P_2=x$.  If A is reducible, then either $P_1$ or $P_2$ is a factor of A.  Notice $P_1$ and $P_2$ are respectively equal to zero when $x=-1$ or $x=0$.  If $A(x)$ is reducible with $P_1$ as a factor, then $A(-1)=0$.  If $A(x)$ is reducible with $P_2$ as a factor, then $A(0)=0$.  Otherwise $A(x)$ is irreducible.\\
\\
Let $A_1(x)=x^3+x+1$, then \\
\tab$A_1(0)=0+0+1=1$ and $A_1(-1)=-1-1+1=-1$\\
\\
Let $A_2(x)=x^3+x^2+1$, then \\
\tab$A_2(0)=0+0+1=1$ and $A_2(-1)=-1+1+1=1$\\
\\
Neither $A_1(x)$ or $A_2(x)$ have $P_1$ or $P_2$ as factors, and are therefore irreducible.\\
\\
2.b.i)\\
Since $x^4+x+1\equiv 0  \pmod{x^4+x+1}$\\
$x^4\equiv x+1  \pmod{x^4+x+1}$\\
$x^5\equiv x^2+x  \pmod{x^4+x+1}$\\
$x^6\equiv x^3+x^2  \pmod{x^4+x+1}$\\

$f(x)g(x)=(x^2+1)(x^3+x^2+1)$\\
$=x^5+x^3+x^4+x^2+x^2+1$\\
$=x^5+x^3+x^4+1$\\
$=(x^2+x)+(x+1)+x^3+1$\\
$=x^3+x^2$\\
\\
2.b.ii)\\
Since $x^4+x+1\equiv 0  \pmod{x^4+x+1}$\\
$x^4+x\equiv 1  \pmod{x^4+x+1}$\\
$x(x^3+1)\equiv 1  \pmod{x^4+x+1}$\\
\\
So given $f(x)=x$, then $f^-1(x)=(x^3+1)$\\
\\
d.i)\\
$y*[ay^3+by^2+cy+d)]$\\
$=ay^4+by^3+cy^2+dy$\\
$=by^3+cy^2+dy+a$ (since $y^4=1$)\\
\\
d.ii)\\
Base cases:\\
$i=0$: $y^i=y^j$ since $0\equiv 0 \pmod 4$\\
$i=1$: $y^i=y^j$ since $1\equiv 1 \pmod 4$\\
$i=2$: $y^i=y^j$ since $2\equiv 2 \pmod 4$\\
$i=3$: $y^i=y^j$ since $3\equiv 3 \pmod 4$\\
\\
Induction Hypothesis:\\
Assume that $y^i=y^j$ for all $i\in \mathbb{Z}$, where $i \geq 0$ such that $j=i \pmod 4$ and $0\geq j \geq 3$.\\
\\
Suppose $4 \leq k$, where $k\in \mathbb{Z}$.\\
Since $(k+1) \geq 0$ and $(k+1) \in \mathbb{Z}$, by the induction hypothesis, we have $y^{k+1}=y^j$ where $j=(k+i)\pmod 4$ and $0\geq j \geq 3$.\\ 
\\
d.iii)\\
Let $ay^3+by^2+cy+d$ represent any 4-byte vector as polynomial.\\
\\
Base case:\\
$i=0$: $y^0(ay^3+by^2+cy+d)=ay^3+by^2+cy+d$ \\
\tab No bytes have been shifted, so this is a circular left shift of 0 bytes. Using the proof from \emph{d.ii} we get that this is a shift of $j=0$ bytes. \\
$i=1$: $y^1(ay^3+by^2+cy+d)$  Using the proofs from d.i and d.ii, this is a left circular shift of $j=1$ bytes.\\
\\
Induction Hypothesis:\\
Assume for $i\geq 0$ that $y^i(ay^3+by^2+cy+d)=ay^{3+i}+by^{2+i}+cy^{1+i}+dy^i$ is a left circular shift of $j$ bytes where $j=i \pmod 4$ and $j\geq 0$.\\
\\
Suppose $k \in \mathbb{Z}$ and $k \geq 2$.\\
%Show $y^{k+1}(ay^3+by^2+cy+d)$ is a left circular shift of k+1 bytes\\
\\
$y^{k+1}(ay^3+by^2+cy+d)= y^k(y^1(ay^3+by^2+cy+d))$\\
%$= y^k(by^3+cy^2+dy+a)$ (a left circular shift of 1 byte)\\
$= y^k(ay^{3+1}+by^{2+1}+cy^{1+1}+dy^{1})$\\
$= ay^{3+(k+1)}+by^{2+(k+1)}+cy^{1+(k+1)}+dy^{k+1}$ (by IH)\\
\\
Since $k+1\geq 0$ and $y^{k+1}=y^j$ where $j=k+1 \pmod 4$ and $0\geq j\geq 3$, the multiplication of any 4-byte vector with $y^{k+1}$ is a left circular shift of $j$ bytes.\\
\\
\end{answer}
\end{problem}

\pbitem Arithmetic with the constant polynomial of MixColumns in AES
\begin{problem}
\begin{answer}
\\
3.a)\\
$c(1)=1$\\
$c(2)=x$\\
$c(3)=x+1$\\
\\
b.i)   
$(01)(b)=1(b_7x^7...+b_1x+b_0$)\\
$d_i=b_i$\\
\\
b.ii)\\
$x^8\equiv x^4+x^3+x+1 \pmod {x^8+x^4+x^3+x+1}$\\
$(02)(b)=x(b_7x^7+...+b_1x+b_0)$\\
$=b_7x^8+b_6x^7...+b_1x^2+b_0x)$\\
$=b_7(x^4+x^3+x+1)+b_6x^7+...+b_1x^2+b_0x)$\\
$d=b_6x^7+b_5x^6+b_4x^5+(b_7+b_3)x^4+(b_7+b_2)x^3+b_1x^2+(b_7+b_0)x+b_7$\\
$d_7=b_6$\\
$d_6=b_5$\\
$d_5=b_4$\\
$d_4=b_7\oplus b_3$\\
$d_3=b_7\oplus b_2$\\
$d_2=b_1$\\
$d_1=b_7\oplus b_0$\\
$d_0=b_7$\\
\\
b.iii)\\
$(03)(b)=(x+1)(b_7x^7+...+b_1x+b_0)$\\
$=(b_7x^8+b_6x^7+...+b_1x^2+b_0x)+(b_7x^7+...+b_1x+b_0)$\\
$=b_7(x^4+x^3+x+1)+(b_6\oplus b_7)x^7+...+(b_0\oplus b_1)x+b_0)$\\
$d_7=b_6\oplus b_7$\\
$d_6=b_5\oplus b_6$\\
$d_5=b_4\oplus b_5$\\
$d_4=b_3\oplus b_4 \oplus b_7$\\
$d_3=b_2\oplus b_3 \oplus b_7$\\
$d_2=b_1\oplus b_2$\\
$d_1=b_0\oplus b_1 \oplus b_7$\\
$d_0=b_0 \oplus b_7$\\

c.i)\\
$y^4\equiv 1 \pmod {y^4+1}$\\
$y^5\equiv y \pmod {y^4+1}$\\
$y^6\equiv y^2 \pmod {y^4+1}$\\
\\
$t(x)=c(y)s(y) \pmod {y^4+1}$\\
$=[(03)y^3+(01)y^2+(01)y+(02)](s_3y^3+s_2y^2+s_1y+s_0) \pmod {y^4+1}$\\
$=(03)(s_3y^6+s_2y^5+s_1y^4+s_0y^3)$\\
$+(01)(s_3y^5+s_2y^4+s_1y^3+s_0y^2)$\\
$+(01)(s_3y^4+s_2y^3+s_1y^2+s_0y)$\\
$+(02)(s_3y^3+s_2y^2+s_1y+s_0)$\\
\\
$=(03)s_3y^6$\\
$+((03)s_2+(01)s_3)y^5$\\
$+((03)s_1+(01)s_2+(01)s_3)y^4$\\
$+((03)s_0+(01)s_1+(01)s_2+(02)s_3)y^3$\\
$+((01)s_0+(01)s_1+(02)s_2)y^2$\\
$+((01)s_0+(02)s_1)y$\\
$+(02)s_0$\\
\\
$=(03)s_3y^2$\\
$+((03)s_2+(01)s_3)y$\\
$+((03)s_1+(01)s_2+(01)s_3)$\\
$+((03)s_0+(01)s_1+(01)s_2+(02)s_3)y^3$\\
$+((01)s_0+(01)s_1+(02)s_2)y^2$\\
$+((01)s_0+(02)s_1)y$\\
$+(02)s_0$\\
\\
$=((03)s_0+(01)s_1+(01)s_2+(02)s_3)y^3$\\
$+((01)s_0+(01)s_1+(02)s_2+(03)s_3)y^2$\\
$+((01)s_0+(02)s_1+(03)s_2+(01)s_3)y$\\
$+((02)s_0+(03)s_1+(01)s_2+(01)s_3)$\\
\\
$t(x)=t_3y^3+t_2y^2+t_1y+t_0$\\
\\
c.ii)\\
\[
C=
\begin{bmatrix}
3 & 1 & 1 & 2 \\
1 & 1 & 2 & 3\\
1 & 2 & 3 & 1\\
2 & 3 & 1 & 1\\
\end{bmatrix}
\]
\\
\end{answer}
\end{problem}

\pbitem Error propagation in block cipher modes
\begin{problem}
\begin{answer}
\\
a)\\
i)\\
ECB: Only $P_i$ is affected since each block is handled independently.\\
ii)\\
CBC: $P_i$ and $P_{i+1}$ are affected since any $C_i$ block only affects the plaintext of the block following it.\\
iii)\\
OFB: Only $P_i$ is affected since OFB does not use $C_i$ in the decryption of $P_{i+1}$ (only during encryption)...or rather it relies on purely on the IV.\\
iv)\\
CFB: $P_i$ to $P_i+k$ are affected since only the last $k$ ciphertext blocks are kept in the register for decrypting.\\
v)\\
CTR: Only $P_i$ is affected $CTR_i$ is simply a counter and is not dependent on the value of $C_i$.\\
\\
b)\\
ECB: Only $P_i$\\
CBC: All of them\\
OFB: All of them\\
CFB: Only $P_i$\\
CTR: Only $P_i$\\
\\
\end{answer}
\end{problem}

\end{problemlist}
\end{document}
