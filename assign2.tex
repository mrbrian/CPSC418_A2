\documentclass{assignment}
\usepackage{amsfonts}
\usepackage{intcalc}
\coursetitle{Introduction to Cryptography}
\courselabel{CPSC 418}
\exercisesheet{Home Work \#2}{}
\student{Brian Yee - 00993104}
\semester{Fall 2016}
\newcommand\tab[1][0.5cm]{\hspace*{#1}}
%\usepackage[pdftex]{graphicx}
%\usepackage{subfigure}

\begin{document}

\begin{center}
\renewcommand{\arraystretch}{2}
\begin{tabular}{|c|c|c|} \hline
Problem & Marks \\ \hline \hline
1 & \\ \hline
2 & \\ \hline
3 & \\ \hline
4 & \\ \hline
5 & \\ \hline
6 & \\ \hline
7 & \\ \hline \hline
Total & \\ \hline
\end{tabular}
\end{center}

\bigskip

\begin{problemlist}
\pbitem Conditional entropy
\begin{problem}
\begin{answer}
\\
1.a)\\

$H(M|C)=\sum_{c\in C}p(C)\sum_{m\in M}p(M|C)log_2(\frac{1}{p(M|C)})$\\

$H(M|C)=\sum p(C)\sum p(M|C)log_2(\frac{1}{p(M|C)})$\\
$=\frac{1}{4}(\frac{1}{2}log_2(2) + \frac{1}{2}log_2(2)) + \frac{1}{4}(\frac{1}{2}log_2(2) + \frac{1}{2}log_2(2)) + \frac{1}{4}(\frac{1}{2}log_2(2) + \frac{1}{2}log_2(2)) + \frac{1}{4}(\frac{1}{2}log_2(2) + \frac{1}{2}log_2(2))$\\
\\
1.b)\\
Since the cryptosystem provides perfect secrecy, $p(x|y)=p(x)$.\\
$\sum p(M)log_2(\frac{1}{p(M)})$\\
$=\frac{1}{|M|}log_2(\frac{1}{p(M)})+\frac{1}{|M|}log_2(\frac{1}{p(M)})+...+\frac{1}{|M|}log_2(\frac{1}{p(M)})$  ($|M|$ total terms)\\
$=|M|*\frac{1}{|M|}log_2(\frac{1}{p(M)})$ \\
$=log_2(\frac{1}{p(M)})$ 

$H(M|C)=\sum p(C)\sum p(M|C)log_2(\frac{1}{p(M|C)})$\\
$=\sum p(C)\sum p(M)log_2(\frac{1}{p(M)})$\\
$=\sum p(C)log_2(\frac{1}{p(M)})$\\

$p(C)=\frac{p(C|M)p(M)}{p(M|C)}$\\
$p(C)=\frac{p(C|M)p(M)}{p(M)}$\\
$p(C)=p(C|M)$\\
...\\
$=\sum p(M)log_2(\frac{1}{p(M)})$\\
$=H(M)$\\
\\
1.c)\\
No, since $p(M|C)=\frac{1}{2}\neq\frac{1}{4}=p(M)$.\\
\\
\end{answer}
\end{problem}

\pbitem Binary polynomial arithmetic
\begin{problem}
\begin{answer}
\\
2.a.i)\\
$x^3$\\
$x^3+1$\\
$x^3+x$\\
$x^3+x+1$\\
$x^3+x^2$\\
$x^3+x^2+1$\\
$x^3+x^2+x$\\
$x^3+x^2+x+1$\\
\\
2.a.ii)\\
$x^3=x*x*x$\\
$x^3+1=(x+1)(x^2-x+1)$\\
$x^3+x=x(x^2+x)$\\
$x^3+x+1=$ irreducible\\
$x^3+x^2=x^2(x+1)$\\
$x^3+x^2+1=$ irreducible\\
$x^3+x^2+x=x(x^2+x+1)$\\
$x^3+x^2+x+1=(x+1)(x^2+1)$\\
\\
2.a.iii)\\

\end{answer}
\end{problem}

\pbitem Arithmetic with the constant polynomial of
MixColumns
in AES
\begin{problem}
\begin{answer}
1.a)\\
Show 1A= A1
and 0A = A0

\end{answer}
\end{problem}

\end{problemlist}
\end{document}
