\documentclass{assignment}
\usepackage[normalem]{ulem}
\usepackage{amsfonts}
\usepackage{amsmath}
\usepackage{intcalc}
\coursetitle{Introduction to Cryptography}
\courselabel{CPSC 418}
\exercisesheet{Home Work \#2}{}
\student{Brian Yee - 00993104}
\semester{Fall 2016}
\newcommand\tab[1][0.5cm]{\hspace*{#1}}
%\usepackage[pdftex]{graphicx}
%\usepackage{subfigure}

\begin{document}

\begin{center}
\renewcommand{\arraystretch}{2}
\begin{tabular}{|c|c|c|} \hline
Problem & Marks \\ \hline \hline
1 & \\ \hline
2 & \\ \hline
3 & \\ \hline
4 & \\ \hline
5 & \\ \hline
6 & \\ \hline
7 & \\ \hline \hline
Total & \\ \hline
\end{tabular}
\end{center}

\bigskip

\begin{problemlist}
\pbitem Conditional entropy
\begin{problem}
\begin{answer}
\\
1.a)\\
$H(M|C)=\sum_{c\in C}p(C)\sum_{m\in M}p(M|C)log_2(\frac{1}{p(M|C)})$\\
$H(M|C)=\sum p(C)\sum p(M|C)log_2(\frac{1}{p(M|C)})$\\
%$=\frac{1}{4}(\frac{1}{2}log_2(2) + \frac{1}{2}log_2(2)) + \frac{1}{4}(\frac{1}{2}log_2(2) + \frac{1}{2}log_2(2)) + \frac{1}{4}(\frac{1}{2}log_2(2) + \frac{1}{2}log_2(2)) + \frac{1}{4}(\frac{1}{2}log_2(2) + \frac{1}{2}log_2(2))$\\
$=4*\frac{1}{4}(\frac{1}{2}log_2(2)+ \frac{1}{2}log_2(2))$\\
$=1$\\
\\
1.b)\\
$H(M|C)=\sum p(C)\sum p(M|C)log_2(\frac{1}{p(M|C)})$\\

Since the cryptosystem provides perfect secrecy, $p(M|C)=p(M)$.\\

$=\sum p(C)\sum p(M)log_2(\frac{1}{p(M)})$\\

We know $\sum p(M)log_2(\frac{1}{p(M)})=log_2(\frac{1}{p(M)})$, when a cryptosystem provides perfect secrecy. \\

$=\sum p(C)log_2(\frac{1}{p(M)})$\\
\\
With perfect secrecy, every M is equiprobable, so every C is equiprobable.
Since $|C| = |M|$  (as every unique C comes from encrypting some unique M), so we have
p(C) = p(M).\\
\\
Thus,\\

$=\sum p(M)log_2(\frac{1}{p(M)})$\\
$=H(M)$\\
\\
1.c)\\
No, since $p(M|C)=\frac{1}{2}\neq\frac{1}{4}=p(M)$.\\
\\
\end{answer}
\end{problem}

\pbitem Binary polynomial arithmetic
\begin{problem}
\begin{answer}
\\
2.a.i)\\
$x^3$\\
$x^3+1$\\
$x^3+x$\\
$x^3+x+1$\\
$x^3+x^2$\\
$x^3+x^2+1$\\
$x^3+x^2+x$\\
$x^3+x^2+x+1$\\
\\
2.a.ii)\\
$x^3=x*x*x$\\
$x^3+1=(x+1)(x^2-x+1)$\\
$x^3+x=x(x^2+x)$\\
$x^3+x+1=$ irreducible\\
$x^3+x^2=x^2(x+1)$\\
$x^3+x^2+1=$ irreducible\\
$x^3+x^2+x=x(x^2+x+1)$\\
$x^3+x^2+x+1=(x+1)(x^2+1)$\\
\\
2.a.iii)\\
Let $A(x)$ be a degree 3 polynomial.  If $A(x)$ is reducible, then it must the product of a degree 1 polynomial and some other polynomial(s) (of degree 2 or 1).  In either case, there is a polynomial of degree 1 as a factor.  The two possible polynomials of degree 1 are $P_1=x+1$ and $P_2=x$.  If A is reducible, then either $P_1$ or $P_2$ is a factor of A.  Notice $P_1$ and $P_2$ are respectively equal to zero when $x=-1$ or $x=0$.  If $A(x)$ is reducible with $P_1$ as a factor, then $A(-1)=0$.  If $A(x)$ is reducible with $P_2$ as a factor, then $A(0)=0$.  Otherwise $A(x)$ is irreducible.\\
\\
Let $A_1(x)=x^3+x+1$, then \\
\tab$A_1(0)=0+0+1=1$ and $A_1(-1)=-1-1+1=-1$\\
\\
Let $A_2(x)=x^3+x^2+1$, then \\
\tab$A_2(0)=0+0+1=1$ and $A_2(-1)=-1+1+1=1$\\
\\
Neither $A_1(x)$ or $A_2(x)$ have $P_1$ or $P_2$ as factors, and are therefore irreducible.\\
\\
2.b.i)\\
Since $x^4+x+1\equiv 0  \pmod{x^4+x+1}$\\
$x^4\equiv x+1  \pmod{x^4+x+1}$\\
$x^5\equiv x^2+x  \pmod{x^4+x+1}$\\
$x^6\equiv x^3+x^2  \pmod{x^4+x+1}$\\

$f(x)g(x)=(x^2+1)(x^3+x^2+1)$\\
$=x^5+x^3+x^4+x^2+x^2+1$\\
$=x^5+x^3+x^4+1$\\
$=(x^2+x)+(x+1)+x^3+1$\\
$=x^3+x^2$\\
\\
2.b.ii)\\
Since $x^4+x+1\equiv 0  \pmod{x^4+x+1}$\\
$x^4+x\equiv 1  \pmod{x^4+x+1}$\\
$x(x^3+1)\equiv 1  \pmod{x^4+x+1}$\\
\\
So given $f(x)=x$, then $f^-1(x)=(x^3+1)$\\
\\
d.i)\\
$y*(ay^3+by^2+cy+d)$\\
$\equiv ay^4+by^3+cy^2+dy$\\
$\equiv by^3+cy^2+dy+a \pmod {y^4+1}$ \tab (since $y^4\equiv 1 \pmod {y^4+1}$)\\
\\
d.ii)\\
Since $1\equiv y^4 \pmod {y^4+1}$,\\
$y^i\equiv y^iy^4\equiv y^{i+4}\equiv y^j \pmod {y^4+1}$ where $j\equiv i+4\equiv i \pmod 4$ and $0\leq j \leq 3$.\\
\\
d.iii)\\
Let $b=ay^3+by^2+cy+d$ represent any 4-byte vector as polynomial.\\
Since we have d.ii, this can be split into 4 cases.\\
\\
Case $y^i\equiv y^0 \pmod {y^4+1}$: \\
\tab$y^i*b=y^0(ay^3+by^2+cy+d)\equiv ay^3+by^2+cy+d \pmod {y^4+1}$ \\
\tab This is a left circular shift of $j=0$ bytes.\\
\\
Case $y^i\equiv y^1 \pmod {y^4+1}$: \\
\tab$y^i*b=y^1(ay^3+by^2+cy+d)$\\
\tab$\equiv by^3+cy^2+dy+a$  \tab (by d.i)\\
\tab This is a left circular shift of $j=1$ bytes.\\
\\
Case $y^i\equiv y^2 \pmod {y^4+1}$: \\
\tab$y^i*b=y^2(ay^3+by^2+cy+d)$\\
\tab$=y^1(y^1(ay^3+by^2+cy+d))$  \\
\tab$\equiv y^1(by^3+cy^2+dy+a)$ \tab (by d.i)\\
\tab$\equiv cy^3+dy^2+ay+b \pmod {y^4+1}$ (by d.i) \\
\tab This is a left circular shift of $j=2$ bytes.\\
\\
Case $y^i\equiv y^3 \pmod {y^4+1}$: \\
\tab$y^i*b=y^3(ay^3+by^2+cy+d)$\\
\tab$=y^1y^1(y^1(ay^3+by^2+cy+d))$  \\
\tab$\equiv y^1(y^1(by^3+cy^2+dy+a))$ \tab (by d.i)\\
\tab$\equiv y^1(cy^3+dy^2+ay+b) $ \tab\tab (by d.i)\\
\tab$\equiv dy^3+ay^2+by+c \pmod {y^4+1}$  (by d.i)\\
\tab This is a left circular shift of $j=3$ bytes.\\
\\
In all cases, it is shown that multiplying any 4-byte vector by $y^i$ ($i\geq 0$) is a circular left shift of j bytes where $j\equiv i \pmod 4$ and $0\leq j \leq 3$ (d.ii).\\
\\
\end{answer}
\end{problem}

\pbitem Arithmetic with the constant polynomial of MixColumns in AES
\begin{problem}
\begin{answer}
\\
3.a)\\
$c(1)=1$\\
$c(2)=x$\\
$c(3)=x+1$\\
\\
b.i)   
$(01)(b)=1(b_7x^7...+b_1x+b_0$)\\
$d_i=b_i$\\
\\
b.ii)\\
$x^8\equiv x^4+x^3+x+1 \pmod {x^8+x^4+x^3+x+1}$\\
$(02)(b)=x(b_7x^7+...+b_1x+b_0)$\\
$=b_7x^8+b_6x^7...+b_1x^2+b_0x)$\\
$=b_7(x^4+x^3+x+1)+b_6x^7+...+b_1x^2+b_0x)$\\
$d=b_6x^7+b_5x^6+b_4x^5+(b_7+b_3)x^4+(b_7+b_2)x^3+b_1x^2+(b_7+b_0)x+b_7$\\
$d_7=b_6$\\
$d_6=b_5$\\
$d_5=b_4$\\
$d_4=b_7\oplus b_3$\\
$d_3=b_7\oplus b_2$\\
$d_2=b_1$\\
$d_1=b_7\oplus b_0$\\
$d_0=b_7$\\
\\
b.iii)\\
$(03)(b)=(x+1)(b_7x^7+...+b_1x+b_0)$\\
$=(b_7x^8+b_6x^7+...+b_1x^2+b_0x)+(b_7x^7+...+b_1x+b_0)$\\
$=b_7(x^4+x^3+x+1)+(b_6\oplus b_7)x^7+...+(b_0\oplus b_1)x+b_0)$\\
$d_7=b_6\oplus b_7$\\
$d_6=b_5\oplus b_6$\\
$d_5=b_4\oplus b_5$\\
$d_4=b_3\oplus b_4 \oplus b_7$\\
$d_3=b_2\oplus b_3 \oplus b_7$\\
$d_2=b_1\oplus b_2$\\
$d_1=b_0\oplus b_1 \oplus b_7$\\
$d_0=b_0 \oplus b_7$\\

c.i)\\
$y^4\equiv 1 \pmod {y^4+1}$\\
$y^5\equiv y \pmod {y^4+1}$\\
$y^6\equiv y^2 \pmod {y^4+1}$\\
\\
$t(y)=c(y)s(y) \pmod {y^4+1}$\\
$=[(03)y^3+(01)y^2+(01)y+(02)](s_3y^3+s_2y^2+s_1y+s_0) \pmod {y^4+1}$\\
$=(03)(s_3y^6+s_2y^5+s_1y^4+s_0y^3)$\\
$+(01)(s_3y^5+s_2y^4+s_1y^3+s_0y^2)$\\
$+(01)(s_3y^4+s_2y^3+s_1y^2+s_0y)$\\
$+(02)(s_3y^3+s_2y^2+s_1y+s_0) \pmod {y^4+1}$\\
\\
$=(03)s_3y^6$\\
$+((03)s_2+(01)s_3)y^5$\\
$+((03)s_1+(01)s_2+(01)s_3)y^4$\\
$+((03)s_0+(01)s_1+(01)s_2+(02)s_3)y^3$\\
$+((01)s_0+(01)s_1+(02)s_2)y^2$\\
$+((01)s_0+(02)s_1)y$\\
$+(02)s_0 \pmod {y^4+1}$\\
\\
$=(03)s_3y^2$\\
$+((03)s_2+(01)s_3)y$\\
$+((03)s_1+(01)s_2+(01)s_3)$\\
$+((03)s_0+(01)s_1+(01)s_2+(02)s_3)y^3$\\
$+((01)s_0+(01)s_1+(02)s_2)y^2$\\
$+((01)s_0+(02)s_1)y$\\
$+(02)s_0 \pmod {y^4+1}$\\
\\
$=((03)s_0+(01)s_1+(01)s_2+(02)s_3)y^3$\\
$+((01)s_0+(01)s_1+(02)s_2+(03)s_3)y^2$\\
$+((01)s_0+(02)s_1+(03)s_2+(01)s_3)y$\\
$+((02)s_0+(03)s_1+(01)s_2+(01)s_3) \pmod {y^4+1}$\\
\\
c.ii)\\
\[
C=
\begin{bmatrix}
3 & 1 & 1 & 2 \\
1 & 1 & 2 & 3\\
1 & 2 & 3 & 1\\
2 & 3 & 1 & 1\\
\end{bmatrix}
\]
\\
\end{answer}
\end{problem}

\pbitem Error propagation in block cipher modes
\begin{problem}
\begin{answer}
\\
Let $C_i$ denote the encryption of the i-th message block, $M_i$.\\
Let $P_i$ denote the plaintext obtained from the decryption of $C_i$.\\
\\
a.i)\\
ECB: Only $P_i$ is affected since the decryption of each block is just a simple block substitution(independent of each other).\\
ii)\\CBC: $P_i$ and $P_{i+1}$ are affected since $P_{i+1}$ is the result of $C_i \oplus C_{i+1}$ but $P_{i+2}$ is the result of  $C_{i+1} \oplus C_{i+2}$(does not depend on $C_i$).\\
iii)\\
OFB: Only $P_i$ is affected since any given decryption state has been generated exclusively from the previous state (which eventually originates from the IV).\\
iv)\\
CFB: All plaintext blocks starting from $P_i$ are affected since the $C_i$ will be added to the register for encrypting $P_{i+1}$ into $C_{i+1}$.  As a result, even if the plaintext blocks after $P_i$ are error-free, they will still be encrypted using ciphertext that was created from the errant plaintext block, $P_i$.\\
v)\\
CTR: Only $P_i$ is affected since $CTR_i$ (the source of the output block that gets XORed with $C_i$) is simply an independent counter value.\\
\\
b)\\
The error happens before any encryption takes place, so the entire plaintext would be encrypted and decrypted normally, as if the plaintext had no errors in it.  Thus, only $M_i$ would be affected.\\
\\
\end{answer}
\end{problem}

\end{problemlist}
\end{document}
